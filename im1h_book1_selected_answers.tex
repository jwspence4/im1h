\documentclass{article}
\usepackage{gensymb, amsmath}

\title{IM1H Book 1 Selected Answers}
\author{Mr. Spence}
\date{June 2025}

\begin{document}

\maketitle

\begin{enumerate}

\item

	\begin{enumerate}
	
 	\item $A_{ABCD} = 25, \ A_{BCEF} = 9$
	
	\item --
	
	\item --
	
	\item $A = 34$
	
	\item $l = \sqrt{34}$
	
	\item --
	
	\end{enumerate}
	
\item $l = 4\sqrt{5}$

\item Yes

\item --

\item --

\item $AB = \sqrt{41}$

\item $l = 5\sqrt{2}$

\item $l = \sqrt{5}$, \ No

\item 12

\item $(12, 2), (2, 2)$

\item No

\item $d = 10\sqrt{2}$

\item

	\begin{enumerate}
	
	\item $C = (5, 0).$ Answers may vary.
	
	\item $D = (5, 1).$ Answers may vary.
	
	\item $x = 5$
	
	\item --
	
	\end{enumerate}
	
\item

	\begin{enumerate}
	
	\item 13, 17, 13, 17
	
	\item --
	
	\end{enumerate}
	
\item

	\begin{enumerate}
	
	\item $AP = BP = 2\sqrt{5}$
	
	\item $(3, 5), (2,2), (4,8).$ Answers may vary.
	
	\item No
	
	\item $y = 3(x-2) + 2$
	
	\end{enumerate}
	
\item $(10,3), (-6, 3)$

\item --

\item

	\begin{enumerate}
	
	\item $(0,0), (6,0).$ Answers may vary.
	
	\item $(0,4), (4,2).$ Answers may vary.
	
	\item $(0,4), (2, 2).$ Answers may vary.
	
	\end{enumerate}
	
\item $AB = BC = \sqrt{10}$

\item $C = (6, 3).$ Infinite. Answers may vary for $C$.

\item $(0,0), (\sqrt{13}, 0)$. Answers may vary.

\item $(0,0), (2, 3)$

\item $(0, 0), (\sqrt{13}, 0), (2 + \sqrt{13}, 3), (\sqrt{13}, 6), (0, 6), (-2, 3)$. Answers may vary.

\item $24 - 12\sqrt{2}, \ 24\sqrt{2} - 24$

\item There are an infinite number of different ways.

\item 208m

\item $AP = BP = 5\sqrt{2}$. \newline
	2 more equidistant points: $Q = (2,2), \ R = (5,3)$. Answers may vary. \newline
	All equidistant points: $y = \frac{1}{3}(x - 2) + 2.$

\item Short leg: $21 - 7\sqrt{5}$ \newline
	Long leg: $42 - 14\sqrt{5}$ \newline
	Hypotenuse: $21\sqrt{5} - 35$
	
\item $\frac{5}{12}$

\item $(0, 5 + 4\sqrt{2}), (0, 5 - 4\sqrt{2})$

\item 

	\begin{enumerate}
	
	\item $(0,0), (4, 1)$. Answers may vary.

	\item No.
	
	\end{enumerate}

\item Yes.

\item

	\begin{enumerate}
	
	\item Yes.
	
	\item $\overline{KL}$
	
	\item $\angle KLM$
	
	\item $\angle BAC$
	
	\item They're congruent.
	
	\end{enumerate}

\item They sum to $90\degree$.

\item It's a right angle.

\item

	\begin{enumerate}
	
	\item --
	
	\item $\frac{b}{a}$ is the negative reciprocal of $\frac{-a}{b}$.
	
	\end{enumerate}
	
\item --

\item --

\item A line with an undefined slope is perfectly vertical while a line with a slope of 0 is perfectly horizontal.

\item $n = \frac{49}{4}$

\item $x = 1$. Answers may vary.

\item $y = 1$. Answers may vary. 

\item They're the same line. $-50x + 30y = 90$. 

\item --

\item No.

\item

	\begin{enumerate}
	
	\item $y = \frac{1}{2} (x - 5) + 5$
	
	\item $4x - 5y = 8$
	
	\end{enumerate}
	
\item Yes.

\item $\left( \frac{15}{8}, \frac{15}{8} \right)$

\item $m = -1$

\item Yes.

\item 

	\begin{enumerate}
	
	\item --
	
	\item $\angle Q$; CPTC
	
	\end{enumerate}
	
\item $\triangle ACT \cong \triangle ION$ \newline
	$\triangle ATC \cong \triangle INO$ \newline
	$\triangle CAT \cong \triangle OIN$ \newline
	$\triangle CTA \cong \triangle ONI$ \newline
	$\triangle TAC \cong \triangle NIO$ \newline
	$\triangle TCA \cong \triangle NOI$
	
\item $\triangle BAL \cong \triangle GEL$ \newline
	$\triangle ELB \cong \triangle ALG$ \newline
	$\triangle GEA \cong \triangle BAE$ \newline
	$\triangle ABG \cong \triangle EGB$
	
\item $\angle ABC$ or $\angle CBA$ or $\angle B$ (different ways of writing the same thing). 

\item $\overline{AB}$

\item

	\begin{enumerate}
	
	\item $PNMRQ$
	
	\item $\angle Q$
	
	\end{enumerate}
	
\item

	\begin{enumerate}
	
	\item $d_{AP} = \sqrt{(x + 1)^2 + ( y - 5)^2}$
	
	\item $d_{BP} = \sqrt{(x - 5)^2 + (y - 2)^2}$
	
	\item $\sqrt{(x + 1)^2 + ( y - 5)^2} = \sqrt{(x - 5)^2 + (y - 2)^2}$
	
	\item $4x - 2y = 1$
	
	\item $(2, 3.5)$
	
	\item $m_{AB} = -\frac{1}{2}; \ m_{P} = 2$
	
	\item --
	
	\end{enumerate}
	
\item 
	
	\begin{enumerate}
		
	\item The distance between $(x, y)$ and $(3, 5)$ is equal to the distance between $(x, y)$ and $(7, -1)$.
		
	\item $2x - 3y = 4$
		
	\end{enumerate}
	
\item 

	\begin{enumerate}
	
	\item --
	
	\item --
	
	\item $(6, 9.5)$
	
	\item $(6.2, 9.8)$
	
	\end{enumerate}
	
\item

	\begin{enumerate}
	
	\item $10x - 8y = -35$
	
	\item $(4.5, 10)$. Answers may vary.	
	
	\item $\overline{PA} = \overline{PB}$

	\end{enumerate}

\end{enumerate}

\end{document}