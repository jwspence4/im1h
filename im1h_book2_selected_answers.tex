\documentclass{article}

\usepackage{gensymb, amsmath}

\title{IM1H Book 2 Selected Answers}
\author{IM1H Dream Team}
\date{\today}

\begin{document}

\maketitle

\begin{enumerate}

\item

	\begin{enumerate}
	
	\item $\angle 2, \angle3, \angle 5, \angle 8$
	
	\item $\angle 1, \angle 4, \angle 6, \angle 7$
	
	\item $\angle 3, \angle 5$ \\
		$\angle 2, \angle 8$
	
	\item $\angle 4, \angle 6$ \\
		$\angle 1, \angle 7$
	
	\item Answers may vary. $\angle 1, \angle 5$
	
	\end{enumerate}
	
\item

	\begin{enumerate}
	
	\item $\angle 2 + \angle 4 = 180 \degree$
	
	\item $\angle 2 + \angle 1 + \angle 3 = 180 \degree$
	
	\item $\angle 4 = \angle 1 + \angle 3$
	
	\item -- 
	
	\item --
	
	\end{enumerate}
	
\item

	\begin{enumerate}
	
	\item If $P$ is not equidistant from the coordinate axes, then $P$ is not on the line $y = x$.
	
	\item Yes. Always.
	
	\end{enumerate}
	
\item --

\item Exactly one

\item --

\item

	\begin{enumerate}
	
	\item $\angle AHK \cong \angle HKD$
	
	\item $\angle AHK \cong \angle EHB$
	
	\item $\angle EHB \cong \angle HKD$
	
	\item If two lines are cut by a transversal such that two corresponding angles are congruent, then then lines are parallel.
	
	\item $\angle KHB + \angle HKD = 180 \degree$
	
	\end{enumerate}
	
\item

	\begin{enumerate}
	
	\item $\overline{RU} \parallel \overline{AT}$
	
	\item None
	
	\item $\overline{RU} \parallel \overline{AT}$ \\
		$\overline{RN} \parallel \overline{OT}$
	
	\item $\overline{RU} \parallel \overline{AT}$ \\
		$\overline{AU} \parallel \overline{NT}$
	
	\item $\overline{AU} \parallel \overline{NT}$
	
	\item None
	
	\item $\overline{AU} \parallel \overline{NT}$
	
	\item None
	
	\end{enumerate}
	
\item --

\item --

\item No. Two lines on the same plane that never intersect.

\item It's constant. No.

\item --

\item 

	\begin{enumerate}
	
	\item $\angle a + \angle b + \angle c = 180 \degree$
	
	\item $\angle x  = \angle a$ \\
		$\angle y = \angle b$
	
	\end{enumerate}
	
\item

	\begin{enumerate}
	
	\item $B(6, 0, 0)$ \\
		$C(6, 3, 0)$ \\
		$D(0, 3, 0)$ \\
		$E(0, 0, 2)$ \\
		$F(6, 0, 2)$ \\
		$H(0, 3, 2)$
		
	\item $\overline{AH} = \sqrt{13}$ \\
		$\overline{AC} = 3\sqrt{5}$ \\
		$\overline{AF} = 2\sqrt{10}$ \\
		$\overline{AG} = 7$
	
	\end{enumerate}
	
\item

	\begin{enumerate}
	
	\item $\overline{FD} \parallel \overline{BC}$ \\
		$\overline{AG} \parallel \overline{CD}$
		
	\item $\overline{HS} \parallel \overline{YO}$ \\
		$\overline{XO} \parallel \overline{SN}$
	
	\end{enumerate}
	
\item

	\begin{enumerate}
	
	\item $0 < x < 110$
	
	\item $81 < x < 143$
	
	\end{enumerate}
	
\item --

\item --

\item --

\item --

\item

	\begin{enumerate}
	
	\item $\overline{AB}: y = -\frac{1}{3}x$ \\
		$\overline{BC}: y = -2x$ 
		
	\item $\overline{KA} = 5$ \\
		$\overline{KB} = 5$ \\
		$\overline{KC} = 5$
		
	\item --
	
	\item --
	
	\item Find the intersection of the perpendicular bisectors of any two side lengths.
	
	\end{enumerate}
	
\item

	\begin{enumerate}
	
	\item $4\sqrt{6}$
	
	\item $4\sqrt{5}$
	
	\end{enumerate}
	
\item 

	\begin{enumerate}
	
	\item $\vec{w} = [7, 6]$
	
	\item $\vec{w} = [-5, 8]$
	
	\end{enumerate}
	
\item

	\begin{enumerate}
	
	\item $\overrightarrow{AB} = [3, 4]$ \\
		$\overrightarrow{BC} = [9, -9]$ \\
		$\overrightarrow{AB} + \overrightarrow{BC} = [12, -5]$
	
	\item $\overrightarrow{AC} = \overrightarrow{AB} + \overrightarrow{BC}$
	
	\end{enumerate}
	
\item

	\begin{enumerate}
	
	\item $\sqrt{17}$
	
	\item $\sqrt{a^2 + b^2 + c^2}$
	
	\end{enumerate}
	
\item $[1, 2, 3]$ \\
	$l = \sqrt{14}$

\item
	
	\begin{enumerate}
	
	\item $4 \times 5 \times 3$
	
	\item $5\sqrt{2}$
	
	\end{enumerate}
	
\item $15 \degree$

\item

	\begin{enumerate}
	
	\item 12
	
	\item 48
	
	\item $y = \frac{1}{3}x$
	
	\item $y = -3x + 24$
	
	\item $F = (\frac{36}{5}, \frac{12}{5})$
	
	\item $\frac{12}{5} \sqrt{10}$
	
	\item 96
	
	\item It's twice the area because it's base times height.
	
	\item $\frac{12}{5} \sqrt{10}$
	
	\end{enumerate}
	
\item

	\begin{enumerate}
	
	\item $y = \frac{3}{7} (x + 2)$
	
	\item $y = -\frac{3}{5} (x - 6)$
	
	\item $G = (\frac{8}{3}, 2)$
	
	\item Yes, $M$, $G$, and $C$ are collinear.
	
	\item --
	
	\end{enumerate}
	
\item --

\item

	\begin{enumerate}
	
	\item $x = 98 \degree$
	
	\item $y = 73 \degree$
	
	\item $w = 108 \degree$
	
	\item $u = 26 \degree$
	
	\end{enumerate}
	
\item

	\begin{enumerate}
	
	\item $x = 0$ \\
		$y = -\frac{2}{3}x + 4$ \\
		$y = x + 4$
		
	\item $(0, 4)$
	
	\end{enumerate}
	
\item --

\end{enumerate}

\end{document}