\documentclass{article}

\usepackage{gensymb, amsmath, makecell}

\title{IM1H Book 2 Selected Answers}
\author{IM1H Dream Team}
\date{\today}

\begin{document}

\maketitle

\begin{enumerate}

\item

	\begin{enumerate}
	
	\item $\angle 2, \angle3, \angle 5, \angle 8$
	
	\item $\angle 1, \angle 4, \angle 6, \angle 7$
	
	\item $\angle 3, \angle 5$ \\
		$\angle 2, \angle 8$
	
	\item $\angle 4, \angle 6$ \\
		$\angle 1, \angle 7$
	
	\item Answers may vary. $\angle 1, \angle 5$
	
	\end{enumerate}
	
\item

	\begin{enumerate}
	
	\item $\angle 2 + \angle 4 = 180 \degree$
	
	\item $\angle 2 + \angle 1 + \angle 3 = 180 \degree$
	
	\item $\angle 4 = \angle 1 + \angle 3$
	
	\item -- 
	
	\item --
	
	\end{enumerate}
	
\item

	\begin{enumerate}
	
	\item If $P$ is not equidistant from the coordinate axes, then $P$ is not on the line $y = x$.
	
	\item Yes. Always.
	
	\end{enumerate}
	
\item --

\item Exactly one

\item --

\item

	\begin{enumerate}
	
	\item $\angle AHK \cong \angle HKD$
	
	\item $\angle AHK \cong \angle EHB$
	
	\item $\angle EHB \cong \angle HKD$
	
	\item If two lines are cut by a transversal such that two corresponding angles are congruent, then then lines are parallel.
	
	\item $\angle KHB + \angle HKD = 180 \degree$
	
	\end{enumerate}
	
\item

	\begin{enumerate}
	
	\item $\overline{RU} \parallel \overline{AT}$
	
	\item None
	
	\item $\overline{RU} \parallel \overline{AT}$ \\
		$\overline{RN} \parallel \overline{OT}$
	
	\item $\overline{RU} \parallel \overline{AT}$ \\
		$\overline{AU} \parallel \overline{NT}$
	
	\item $\overline{AU} \parallel \overline{NT}$
	
	\item None
	
	\item $\overline{AU} \parallel \overline{NT}$
	
	\item None
	
	\end{enumerate}
	
\item --

\item --

\item No. Two lines on the same plane that never intersect.

\item It's constant. No.

\item --

\item 

	\begin{enumerate}
	
	\item $\angle a + \angle b + \angle c = 180 \degree$
	
	\item $\angle x  = \angle a$ \\
		$\angle y = \angle b$
	
	\end{enumerate}
	
\item

	\begin{enumerate}
	
	\item $B(6, 0, 0)$ \\
		$C(6, 3, 0)$ \\
		$D(0, 3, 0)$ \\
		$E(0, 0, 2)$ \\
		$F(6, 0, 2)$ \\
		$H(0, 3, 2)$
		
	\item $\overline{AH} = \sqrt{13}$ \\
		$\overline{AC} = 3\sqrt{5}$ \\
		$\overline{AF} = 2\sqrt{10}$ \\
		$\overline{AG} = 7$
	
	\end{enumerate}
	
\item

	\begin{enumerate}
	
	\item $\overline{FD} \parallel \overline{BC}$ \\
		$\overline{AG} \parallel \overline{CD}$
		
	\item $\overline{HS} \parallel \overline{YO}$ \\
		$\overline{XO} \parallel \overline{SN}$
	
	\end{enumerate}
	
\item

	\begin{enumerate}
	
	\item $0 < x < 110$
	
	\item $81 < x < 143$
	
	\end{enumerate}
	
\item --

\item --

\item --

\item --

\item

	\begin{enumerate}
	
	\item $\overline{AB}: y = -\frac{1}{3}x$ \\
		$\overline{BC}: y = -2x$ 
		
	\item $\overline{KA} = 5$ \\
		$\overline{KB} = 5$ \\
		$\overline{KC} = 5$
		
	\item --
	
	\item --
	
	\item Find the intersection of the perpendicular bisectors of any two side lengths.
	
	\end{enumerate}
	
\item

	\begin{enumerate}
	
	\item $4\sqrt{6}$
	
	\item $4\sqrt{5}$
	
	\end{enumerate}
	
\item 

	\begin{enumerate}
	
	\item $\vec{w} = [7, 6]$
	
	\item $\vec{w} = [-5, 8]$
	
	\end{enumerate}
	
\item

	\begin{enumerate}
	
	\item $\overrightarrow{AB} = [3, 4]$ \\
		$\overrightarrow{BC} = [9, -9]$ \\
		$\overrightarrow{AB} + \overrightarrow{BC} = [12, -5]$
	
	\item $\overrightarrow{AC} = \overrightarrow{AB} + \overrightarrow{BC}$
	
	\end{enumerate}
	
\item

	\begin{enumerate}
	
	\item $\sqrt{17}$
	
	\item $\sqrt{a^2 + b^2 + c^2}$
	
	\end{enumerate}
	
\item $[1, 2, 3]$ \\
	$l = \sqrt{14}$

\item
	
	\begin{enumerate}
	
	\item $4 \times 5 \times 3$
	
	\item $5\sqrt{2}$
	
	\end{enumerate}
	
\item $15 \degree$

\item

	\begin{enumerate}
	
	\item 12
	
	\item 48
	
	\item $y = \frac{1}{3}x$
	
	\item $y = -3x + 24$
	
	\item $F = (\frac{36}{5}, \frac{12}{5})$
	
	\item $\frac{12}{5} \sqrt{10}$
	
	\item 96
	
	\item It's twice the area because it's base times height.
	
	\item $\frac{12}{5} \sqrt{10}$
	
	\end{enumerate}
	
\item

	\begin{enumerate}
	
	\item $y = \frac{3}{7} (x + 2)$
	
	\item $y = -\frac{3}{5} (x - 6)$
	
	\item $G = (\frac{8}{3}, 2)$
	
	\item Yes, $M$, $G$, and $C$ are collinear.
	
	\item --
	
	\end{enumerate}
	
\item --

\item

	\begin{enumerate}
	
	\item $x = 98 \degree$
	
	\item $y = 73 \degree$
	
	\item $w = 108 \degree$
	
	\item $u = 26 \degree$
	
	\end{enumerate}
	
\item

	\begin{enumerate}
	
	\item $x = 0$ \\
		$y = -\frac{2}{3}x + 4$ \\
		$y = x + 4$
		
	\item $(0, 4)$
	
	\end{enumerate}
	
\item --

\item

	\begin{enumerate}
	
	\item --
	
	\item --
	
	\item $K$ is equidistant from all three vertices.
	
	\item Yes, they are.
	
	\end{enumerate}
	
\item $x = \pm 6$

\item

	\begin{enumerate}
	
	\item --
	
	\item By C.P.C.T.C., $\overline{DP} \cong \overline{DQ}$, so $D$ is equidistant from $\overline{AB}$ and $\overline{BC}$. Since we showed this for an arbitrary point $D$ on the angle bisector, it must be true for any point on the angle bisector.
	
	\end{enumerate}
	
\item $\angle ABC = 68 \degree$ \\
	$\angle BCA = 56 \degree$
	
\item --

\item

	\begin{gather*}
	a = 124 \degree \\
	b = 56 \degree \\
	c = 56 \degree \\
	d = 38 \degree \\
	e = 38 \degree \\
	f = 76 \degree \\
	g = 66 \degree \\
	h = 104 \degree \\
	k = 76 \degree \\
	n = 86 \degree
	\end{gather*}
	
\item

	\begin{enumerate}
	
	\item --
	
	\item $\overline{AB} \parallel \overline{CD}$ by A.I.A.T.
	
	\item --
	
	\item $\overline{AD} \parallel \overline{BC}$ by A.I.A.T.
	
	\item If both pairs of opposite sides in a quadrilateral are congruent, then the quadrilateral is a parallelogram.
	
	\item If a quadrilateral is a parallelogram, then both pairs of opposite sides are congruent.
	
	\end{enumerate}
	
\item If a quadrilateral is a parallelogram, then its diagonals bisect each other.

\item $3 = \sqrt{x^2 + y^2 + z^2}$ or $9 = x^2 + y^2 + z^2$. The configuration of all such points is a sphere centered at the origin with a radius of 3.

\item --

\item --

\item The sum of the three exterior angles is $360 \degree$.

\item If both pairs of opposite angles in a quadrilateral are congruent, the quadrilateral is a parallelogram.

\item If all adjacent angles in a quadrilateral are supplementary, then the quadrilateral is a parallelogram. 

\item

	\begin{enumerate}
	
	\item $107 \degree$
	
	\item $107 \degree$
	
	\item $107 \degree$
	
	\end{enumerate}
	
\item

	\begin{enumerate}
	
	\item $D = (1, -2)$
	
	\item It appears to be a parallelogram.
	
	\item If a quadrilateral has a pair of opposite, congruent, and parallel sides, it's a parallelogram
	
	\end{enumerate}
	
\item --

\item --

\item

	\begin{gather*}
	\vec{u} + \vec{v} = [6, -1] \\
	\vec{u} - \vec{v} = [4, 5] \\
	\vec{u} + 2\vec{v} = [7, -4] \\
	2\vec{u} - 3\vec{v} = [7, 13]
	\end{gather*}
	
\item $\overline{AB} \parallel \overline{DC}$ \\
	$\overline{AD} \cong \overline{BC}$
	
\item --

\item $(90 - \frac{r}{2}) \degree$

\item No, they intersect in a way that creates one acute angle and one obtuse angle. In \#57, we saw that those angles are $(90 - \frac{r}{2}) \degree$ and $(90 + \frac{r}{2}) \degree$. The only way for them to intersect perpendicularly is for $r$ to equal $0 \degree$ which is impossible in a triangle.

\item --

\item --

\item --

\item

	\begin{enumerate}
	
	\item $\overrightarrow{BA} = -\vec{u}$
	
	\item $\overrightarrow{CA} = -\vec{u} - \vec{v}$
	
	\item $\overrightarrow{MN} = \frac{1}{2} \vec{u} + \frac{1}{2} \vec{v}$
	
	\end{enumerate}
	
\item $[5, -3, 6] \cdot [3, 7, 1] = 15 - 21 + 6 = 0$

\item

	\begin{enumerate}
	
	\item $\overrightarrow{AC} = \vec{u} + \vec{v}$
	
	\item $\overrightarrow{CX} = -\frac{1}{2} \vec{u} - \frac{1}{2} \vec{v}$
	
	\item $\overrightarrow{DB} = \vec{u} - \vec{v}$
	
	\item $\overrightarrow{XM} = \frac{1}{2}\vec{v}$
	
	\end{enumerate}
	
\item

	\begin{enumerate}
	
	\item $\overrightarrow{PS} = [c, d]$ \\
		$\overrightarrow{PQ} = [a, b]$
		
	\item $R = (a + c, b + d)$
	
	\end{enumerate}
	
\item A $3 \times 4$ rectangle.

\item $C = (10, 5)$ \\
	$C = (-4, 3)$
	
\item Yes, there is. Draw diagonals $\overline{AC}$ and $\overline{PR}$. You can show that they create two pairs of congruent triangles across the two quadrilaterals.

\item $[5, -3, 6] \cdot [3, 7, 1] = 15 - 21 + 6 = 0$

\item $[4, 0, -5]$

\item $\left[ \frac{7}{\sqrt{10}}, -\frac{21}{\sqrt{10}}, 0 \right]$

\item --

\item

	\begin{enumerate}
	
	\item $x = 50 \degree$ \\
		$y = 95 \degree$
		
	\item $x = 60 \degree$ \\
		$y = 140 \degree$
		
	\item $x = 30 \degree$ \\
		$y = 30 \degree$
		
	\item $x = 9$
	
	\item $x = 7$ \\
		$y = -3$
		
	\item $x = 6$ \\
		$y = 4$
		
	\item $x = 50 \degree$
	
	\item $x = 51 \degree$ \\
		$y = 112 \degree$
		
	\item $x = 30$ \\
		$y = 10$
		
	\item $x = 6$
	
	\item $x = 50 \degree$
	
	\item $x = -1, 10$
	
	\end{enumerate}
	
\item

	\begin{gather*}
	\angle 1 = 80 \degree \\
	\angle 2 = 165 \degree \\
	\angle 3 = 15 \degree \\
	\angle 4 = 25 \degree \\
	\angle 5 - 75 \degree \\
	\angle 6 = 65 \degree \\
	\angle 7 = 40 \degree \\
	\angle 8 = 140 \degree \\
	\angle 9 = 100 \degree \\
	\angle 10 = 80 \degree
	\end{gather*}
	
\item

	\begin{enumerate}
	
	\item $\overrightarrow{AB} + \overrightarrow{BC}$
	
	\item $\overrightarrow{AB} + \overrightarrow{AD}$
	
	\item $-\overrightarrow{AB} + \overrightarrow{AD}$
	
	\end{enumerate}
	
\item $\overrightarrow{PQ} = -\vec{v} + \vec{w}$ \\
	$\overrightarrow{BC} = -3\vec{v} + 3\vec{w}$
	
\item

	\begin{enumerate}
	
	\item $\vec{a} + \vec{c}$
	
	\item $-\vec{b} + \vec{c}$
	
	\item $\vec{a} + \vec{b} + \vec{c}$
	
	\item $-\vec{a} + \vec{b} + \vec{c}$
	
	\end{enumerate}
	
\item $22 \degree$

\item $90 \degree$
	
\item --

\item --

\item 

	\begin{enumerate}
	
	\item --
	
	\item --
	
	\item Yes
	
	\end{enumerate}
	
\item

	\begin{enumerate}
	
	\item 60
	
	\item 30
	
	\item 20
	
	\end{enumerate}
	
\item No

\item $36 \degree$

\item $36 \degree - 36 \degree - 108 \degree$ \\
	$45 \degree - 45 \degree - 90 \degree$ \\
	$36 \degree - 72 \degree - 72 \degree$
	
\item

	\begin{enumerate}
	
	\item $360 \degree$
	
	\item \phantom{x}
	
		\begin{center}
		\begin{tabular}{| c | c | c | c | c | c |}
		\hline
		Polygon & Sketch & \makecell{Number of\\sides} & \makecell{Number of\\diagonals\\from 1\\vertex} & \makecell{Number of\\triangles} & \makecell{Interior\\angle sum} \\
		\hline
		Triangle & -- & 3 & 0 & 1 & $180\degree$ \\
		\hline
		Quadrilateral & -- & 4 & 1 & 2 & $360 \degree$ \\
		\hline
		Pentagon & -- & 5 & 2 & 3 & $540 \degree$ \\
		\hline
		Hexagon & -- & 6 & 3 & 4 & $720 \degree$ \\
		\hline
		Heptagon & -- & 7 & 4 & 5 & $900 \degree$ \\
		\hline
		Octagon & -- & 8 & 5 & 6 & $1,080 \degree$ \\
		\hline
		Decagon & -- & 10 & 7 & 8 & $1,440 \degree$ \\
		\hline
		Dodecagon & -- & 12 & 9 & 10 & $1,800 \degree$ \\
		\hline
		$n$-gon & -- & $n$ & $n - 3$ & $n-2$ & $180 \degree (n - 2)$ \\
		\hline
		\end{tabular}
		\end{center}
	
	\end{enumerate}
	
\item

	\begin{enumerate}
	
	\item Multiply $180\degree$ by the number of sides minus 2.
	
	\item $3,240\degree$
	
	\end{enumerate}
	
\item $75 \degree$

\item --

\item

	\begin{enumerate}
	
	\item	$\overrightarrow{BC} = \vec{v} - \vec{u}$ \\
		$\overrightarrow{MN} = \frac{1}{2}\vec{v} - \frac{1}{2}\vec{u}$
		
	\item $\overrightarrow{MN} = \frac{1}{2} \overrightarrow{BC} \implies \overline{MN} \parallel \overline{BC}$
	
	\end{enumerate}

\item $360\degree$

\item
	
	\begin{enumerate}
	
	\item $(12, 4, 3)$
	
	\item $\left( \frac{60}{13}, \frac{20}{13}, \frac{15}{13} \right)$
	
	\end{enumerate}
	
\item --

\item $90 \degree$

\item

	\begin{enumerate}
	
	\item $30 \degree$
	
	\item Yes
	
	\end{enumerate}

\item $x = \frac{240}{7}$

\item --

\item One

\item

	\begin{enumerate}
	
	\item $x = 8$ \\
		$y = 10$ \\
		$z = 10$
	
	\item $x = 10$
	
	\item $x = 60\degree$ \\
		$y = 140\degree$
	
	\item $x = 6$ \\
		$y = \frac{13}{2}$
		
	\item $z = 65 \degree$
	
	\item $x = 40$
	
	\end{enumerate}

\end{enumerate}

\end{document}